\subsection{Scale-Invariant Feature Transform (SIFT)}
David G. Lowe hat mit der Einführung des SIFT-Algorithmus einen für die
Bildverarbeitung wertvollen Beitrag geleistet. Der SIFT-Algoritmus findet lokale
Merkmale in Bildern. Diese Merkmale sind invariant gegenüber Bildskalierung und
-drehung und teilweise invariant gegenüber Änderungen der Beleuchtung sowie des
Betrachtungswinkels. Die Wahrscheinlichkeit einer Störung durch Verdeckung
einzelner Elemente oder auftretenden Bildrauschens wird, durch räumliche
Streuung und einem vielseitigen Frequenzbereich, erreicht. Im Folgenden wird die
grobe Vorgehensweise zur Findung der lokalen Merkmale erläutert. 

\begin{enumerate}
    \item \textbf{Scale-space extrema detection}\newline
    Zu Beginn wird aus einem Bild durch repetetives Unschärfen (mittels
    Gaußschen Unschärfefilter) und Halbieren der Größe eine Serie an Bildern
    erzeugt. Auf die Bilder der Bilderserie folgt die Anwendung der Gaußschen
    Differenzfunktion.
    \item \textbf{Keypoint localization}\newline
    Aus den vorher differenzierten Grafiken werden jetzt mit dem
    Marr-Hildreth-Operator interessante Schlüsselpunkte bzw. Merkmale 
    herausgefiltert. Eine dem Harris-Corner Detector ähnliche Prozedur wird
    angewandt, um Punkte auf Linien zwischen Ecken und Punkte mit wenig
    Kontrast (uninteressante Punkte) herauszusieben.
    \item \textbf{Orientation assignment}\newline
    Mithilfe eines Histogramms kann den Schlüsselpunkten nun eine Orientierung
    zugewiesen werden. Dadurch werden die Punkte neben der Robustheit gegenüber
    Skalierung und Translation auch gegen Rotation invariant.
    \item \textbf{Keypoint descriptor}\newline
    Zuletzt wird nach Anwendung einiger mathematischer Verfahren eine Art
    eindeutiger Fingerabdruck des Merkmals berechnet. Schließlich wird dadurch
    die Stabilität gegenüber begrenzter lokaler Formverzerrungen und
    Beleuchtungsänderungen erreicht.
\end{enumerate}

Nach Anwendung dieses Algorithmus muss man bei einem 500x500 Pixel großen Bild
mit in etwa 2000 robusten Merkmalen rechnen. Werden die gefundenen Merkmale in
einer Datenbank gespeichert, können sie schließlich in nahezu Echtzeit auf die
Schlüsselpunkte eines Suchbilds verglichen werden.

Ein ähnlicher patentierter Algorithmus zur Erkennung von Bildmerkmalen ist der
\glqq{}Speeded up robust features (SURF)\grqq{}. Ferner gibt es viele ähnliche
frei verfügbare Algorithmen der Gruppe \glqq{}Bildmerkmalsfindung\grqq{}.