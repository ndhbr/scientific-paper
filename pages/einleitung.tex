\section{Einleitung}\label{einleitung}
Visuelle Ähnlichkeitsbestimmung von Bildern spielt in der heutigen Zeit eine
immer wichtigere Rolle. Anders als bei der Klassifikation von Bildern, geht es
dabei nicht um die korrekte Klassenzuordnung des Bildes, sondern um die konkrete 
visuelle Ähnlichkeit zwischen zwei Bildern \parencite{intro-classification}. 
Inhaltsbasierte Bildvergleichsalgorithmen berücksichtigen Parameter, wie Farbe,
Form, Textur und Struktur. Aus diesen Informationen wird die Überschneidung und
somit die Ähnlichkeit von Bildern bestimmt \parencite{intro-cibr}.

Inhaltsbasierte Algorithmen werden neben der umgekehrten Bildersuche
auch beispielsweise für Dublettenerkennung, Anwendung des
Urheberrechts-Dienstean"-bieter-Gesetzes, Qualitätsprüfung und vieles mehr
verwendet. Da diese Problematik noch immer ein offenes Forschungsthema ist und
es bisher keine allgemeingültige Lösung gibt, beschäftigt sich dieses Paper mit
einer Übersicht aller aktuell gängigen Kategorien von inhaltsbasierten
Bildvergleichsalgorithmen.