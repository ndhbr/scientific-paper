\subsection{Siamese Network (Deep Learning)}
Zu guter Letzt liefert der Image Matching Algorithmus auf Basis des Siamesischen
Neuronalen Netzwerks die besten Ergebnisse. Dabei besteht das neuronale Netzwerk
aus zwei identischen Subnetzwerken. Meist kommt hierfür das sehr verbreitete
Convolutional Neural Network als Subnetzwerk zum Einsatz, welches die
Merkmalsvektoren aus den Bildern extrahiert. \parencite{siamese-orig-paper}

Die beiden entstehenden Vektoren dienen als Eingabe für die Contrastive Loss
Function. Dieser Vorgang minimiert die euklidische Distanz der Vektoren im Falle
eines Matches und maximiert diese bei inkomparabelen Bilderpaaren.
\parencite{siamese-orig-paper}

Wie bei vielen Machine Learning Ansätzen ist bei der Verwendung einer der
größten Nachteile das Training. Damit das Siamese Network optimal funktioniert,
muss das Netzwerk mit einem sehr großen Datensatz an klassifizierten
Bilderpaaren trainiert werden.
\parencite{mse-naive-approach}