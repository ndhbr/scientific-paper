\subsection{Siamese Network (Deep Learning)}
Zu guter Letzt liefert der Image Matching Algorithmus auf Basis des Siamesischen
Neuronalen Netzwerks die besten Ergebnisse. Dabei besteht das neuronale Netzwerk
aus zwei identischen Subnetzwerken. Meist kommt hierfür das sehr verbreitete
\textit{Convolutional Neural Network}\footnote{https://www.ibm.com/cloud/learn/convolutional-neural-networks [09.12.2022]}
als Subnetzwerk zum Einsatz, welches die Merkmalsvektoren aus den Bildern
extrahiert. \parencite{siamese-orig-paper}

Die beiden entstehenden Vektoren dienen als Eingabe für die
\textit{Contrastive Loss Function}\footnote{https://towardsdatascience.com/contrastive-loss-explaned-159f2d4a87ec [09.12.2022]}.
Im Falle eines Matches minimiert dieser Vorgang die euklidische Distanz,
andernfalls maximiert er die Distanz bei inkomparabelen Bilderpaaren.
\parencite{siamese-orig-paper}

Wie bei vielen Machine Learning Ansätzen ist bei der Verwendung einer der
größten Nachteile das Training. Damit das Siamese Network optimal funktioniert,
muss das Netzwerk mit einem sehr großen Datensatz an klassifizierten
Bilderpaaren trainiert werden \parencite{mse-naive-approach}. Das ist außerdem
der Grund, wieso im Rahmen dieser Abhandlung keine Anwendung an Testbildern 
durchgeführt werden konnte.
