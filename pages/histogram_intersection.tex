\subsection{Histogram Intersection}
Ein robusteres, aber auch rechenaufwendigeres Verfahren ist der Histogram
Intersection Algorithmus von Swain und Ballard. Histogramme sind stabil
gegenüber Verschiebungen und Rotationen um die Betrachtungsachse. Außerdem
ändern sie sich bei Änderungen des Blickwinkels, des Maßstabs und bei Verdeckung
von Elementen nur langsam. Gegen Belichtungsänderungen können stark reduzierte
Farbhistogramme jedoch Probleme bereiten. \parencite{histogram-swain-ballard}

Im Standardfall wird die Histogram Intersection auf Farbhistogramme angewandt. 
Zuerst muss die Menge an zu diskretisierenden Farben im Histogramm festgelegt
werden - zum Beispiel 200. Folglich gibt es 200 mögliche \glqq{}Farbeimer\grqq{}
in die wir jeden Pixel der Bilder einsortieren. Mit den durch das Referenz- und
das Suchbild berechneten Histogrammen, kann schließlich die Überschneidung
festgestellt werden. Je stärker sich die Histogramme überschneiden, desto
ähnlicher sind die Bilder. \parencite{histogram-image-similarity} Zusätzlich
wird wie beim Hashing ein gewisser Schwellenwert vorher definiert. Der Vorgang
kann beispielsweise auch auf die einzelnen Farbkanäle aufgeteilt werden. Bei RGB
würde das bedeuten, dass man jeweils den Rot-, Grün-, und Blau-Kanal einzeln
betrachtet. \parencite{histogram-swain-ballard}

Weitere Untersuchungen haben ergeben, dass sowohl die Wahl des Farbraums als
auch die Festlegung des Quantization Levels (Anzahl der Farbeimer) eine große
Rolle bei der Erfolgsschance dieses Vorgehensmodells spielen. Hierbei sorgte
wohl der CIELab-Farbraum im Allgemeinen für die besten Ergebnisse.
\parencite{histogram-image-similarity}

Eine zusätzliche Möglichkeit die Genauigkeit des Algorithmus zu verbessern ist
das Hinzufügen und Vergleichen anderer Histogramme. Ein mögliches Beispiel wäre
ein \glqq{}Texture Direction\grqq{} oder \glqq{}Texture Scale\grqq{} Histogramm.
In diesen Fällen wird nicht versucht anhand der Farben eine Ähnlichkeit zu
erkennen, sondern durch die Struktur des Bildes. Hierbei könnten beispielsweise
aus dem Bild extrahierte Ecken oder Kanten in ein Histogramm sortiert werden und
schließlich auf Überschneidungen analysiert werden.
\parencite{histogram-stackoverflow} Offensichtlich wird jedoch jeder weitere
zusätzlicher Berechnungsschritt die allgemeine Performance des Systems
beeinträchtigen.