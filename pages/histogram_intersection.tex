\subsection{Histogram Intersection}
Ein robusteres, aber auch rechenaufwendigeres Verfahren ist der Histogram
Intersection Algorithmus von Swain und Ballard. Histogramme sind stabil
gegenüber Translation und Rotation um die Betrachtungsachse. Außerdem ändern
sie sich bei Änderungen des Blickwinkels, des Maßstabs und bei Verdeckung von
Elementen nur langsam. Gegen Belichtungsänderungen können stark reduzierte
Farbhistogramme jedoch Probleme bereiten. 
% Quelle: Swain and Ballard P.3

Im Standardfall wird die Histogram Intersection auf Farbhistogramme angewandt. 
Zuerst muss die Menge an zu diskretisierenden Farben im Histogramm festgelegt
werden - zum Beispiel 100. Folglich gibt es 100 mögliche \glqq{}Farbeimer\grqq{}
in die wir jeden Pixel der Bilder einsortieren. Mit den durch das Referenz- und
das Suchbild berechneten Histogrammen, kann schließlich die Überschneidung
festgestellt werden. Je stärker sich die Histogramme überschneiden, desto
ähnlicher sind die Bilder. Zusätzlich wird wie beim Hashing ein gewisser
Schwellenwert vorher definiert. Der Vorgang kann beispielsweise auch auf die
einzelnen Farbkanäle aufgeteilt werden. Bei RGB würde das bedeuten, dass man
jeweils den Rot-, Grün-, und Blau-Kanal einzeln betrachtet. 

Weitere Untersuchungen haben ergeben, dass sowohl die Wahl des Farbraums als
auch die Festlegung des Quantization Levels (Anzahl der Farbeimer) eine große
Rolle bei der Erfolgsschance dieses Vorgehensmodells spielen. Hierbei sorgte
wohl der CIELab-Farbraum im Allgemeinen für die besten Ergebnisse.

Eine zusätzliche Möglichkeit die Genauigkeit des Algorithmus zu verbessern ist
das Hinzufügen eines sogenannten \glqq{}Texture Direction\grqq{}-Histogramms. In
diesem Fall wird durch eine Eckenerkennung die Struktur und Richtungen des
Bildes bestimmt. Diese werden wie bei den Farbhistogrammen auch in einem
Histogramm mit vorher definierten Bins einsortiert. Abschließend kann, wie oben,
beschrieben die Überschneidung und somit die Ähnlichkeit der Struktur zweier
Bilder mithilfe des \glqq{}Histogram Intersection\grqq{}-Algorithmus von Swain
und Ballard berechnet werden. Offensichtlich wird jedoch jeder weitere 
zusätzlicher Berechnungsschritt die allgemeine Performance des Systems
beeinträchtigen.