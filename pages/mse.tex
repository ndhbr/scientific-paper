\subsection{Mean squared error (MSE)}
Der naive Ansatz zwei Bilder auf visuelle Ähnlichkeit zu bewerten ist es,
iterativ alle Pixelintensitäten miteinander zu vergleichen. Als Werkzeug zum
Vergleich dient hierbei die mittlere quadratische Abweichung (Mean squared
error) \parencite{mse-overview}. Je niedriger die Abweichung zwischen den
Bildern, desto mehr stimmen sie überein. Obwohl dieses Verfahren verhältnismäßig
leicht zu implementieren ist, ist es nur für Fälle geeignet, bei denen sowohl
das Referenz- als auch das Testbild fast identisch sind. Bereits kleinste
Helligkeitsänderungen bei gleichbleibendem Inhalt ergeben nach Berechnung des
euklidischen Abstands eine hohe mittlere quadratische Abweichung (siehe
Abbildung \ref{fig:mse-naive}). \parencite{mse-naive-approach}

\begin{figure}[H]
    \centering
    \includegraphics[width=\textwidth]{mse-naive-approach}
    \caption{MSE: Anwendung an verschiedenen Testbildern}
    \label{fig:mse-naive}
    \bildquelle{Eigene Darstellung}
\end{figure}