% Festlegung des allgemeinen Dokumentenformats
\documentclass[a4paper,12pt]{article}

% Schrift
\usepackage[T1]{fontenc}
\usepackage{lmodern}
\usepackage[utf8]{inputenc}
\usepackage[ngerman]{babel}

% Bilder
\usepackage{graphicx}
\usepackage{float}
\graphicspath{{./assets/img}}

% Variablen
% Variablen
\newcommand{\titleDocument}{Inhaltsbasierte Bildvergleiche}
\newcommand{\subjectDocument}{Möglichkeiten zur Bestimmung der visuellen
Ähnlichkeit zwischen zwei Bildern}
\newcommand*{\bildquelle}{
  \footnotesize Quelle:
}
\newcommand{\code}[1]{\noindent\ignorespaces\texttt{#1}}

% Code
\usepackage{listings}
\usepackage{color}

\definecolor{black}{rgb}{0,0,0}
\definecolor{dkgreen}{rgb}{0,0.6,0}
\definecolor{gray}{rgb}{0.5,0.5,0.5}
\definecolor{mauve}{rgb}{0.58,0,0.82}

\lstset{literate=%
    {Ö}{{\"O}}1
    {Ä}{{\"A}}1
    {Ü}{{\"U}}1
    {ß}{{\ss}}1
    {ü}{{\"u}}1
    {ä}{{\"a}}1
    {ö}{{\"o}}1
    {~}{{\textasciitilde}}1
}

\lstdefinestyle{Bash} {
  frame=single,
  language=Bash,
  aboveskip=3mm,
  belowskip=3mm,
  showstringspaces=false,
  columns=flexible,
  basicstyle={\small\ttfamily},
  numbers=none,
  numberstyle=\tiny\color{black},
  keywordstyle=\color{black},
  commentstyle=\color{black},
  stringstyle=\color{black},
  breaklines=true,
  breakatwhitespace=true,
  tabsize=3
}

\lstset{emph={\$},emphstyle=\textbf}

\lstdefinestyle{Python} {
  frame=single,
  language=Bash,
  aboveskip=3mm,
  belowskip=3mm,
  showstringspaces=false,
  columns=flexible,
  basicstyle={\small\ttfamily},
  numbers=left,
  numberstyle=\tiny\color{mauve},
  keywordstyle=\color{mauve},
  commentstyle=\color{dkgreen},
  stringstyle=\color{dkgreen},
  breaklines=true,
  breakatwhitespace=true,
  tabsize=2
}

% mehrseitige Tabellen ermöglichen
\usepackage{longtable}
\usepackage{diagbox}

% Packet für Seitenrandabstände und Einstellung für Seitenränder
\usepackage{geometry}
% Internet
%\geometry{left=3.5cm, right=2.5cm, top=2.5cm, bottom=2cm}
% Kern
\geometry{a4paper, top=27mm, left=20mm, right=20mm, bottom=35mm, headsep=10mm,
footskip=12mm}

% bricht lange URLs "schön" um
\usepackage[hyphens,obeyspaces,spaces]{url}

% Festlegung Art der Zitierung
\usepackage{csquotes}
\usepackage[style=apa, backend=biber]{biblatex}
\addbibresource{assets/literatur.bib}

% Abstand zwischen Absätze
\setlength{\parindent}{2em}
\setlength{\parskip}{1em}

% Paket für Zeilenabstand
\usepackage{setspace}
\onehalfspacing
% Kern:
% \setstretch{1.15}

% für Bildbezeichner
\usepackage{capt-of}

% für Stichwortverzeichnis
\usepackage{makeidx}

% für Abkürzungsverzeichnis
\usepackage{acronym}

% Für Phantomsection
\usepackage{hyperref}

% Für Tabellen
\usepackage{tabularx}

% Konfiguriere das Inhaltsverzeichnis
\usepackage{tocbasic}

% Anhang: PDF einfügen
\usepackage{pdfpages}

% subsubsubsection durch paragraph
\usepackage{titlesec}
\setcounter{secnumdepth}{4}
\titleformat{\paragraph}
{\normalfont\normalsize\bfseries}{\theparagraph}{1em}{}
\titlespacing*{\paragraph}
{0pt}{3.25ex plus 1ex minus .2ex}{1.5ex plus .2ex}
% Kern:
\titlespacing{\section}{0pt}{12pt plus 4pt minus 2pt}{8pt plus 2pt minus 2pt}
\titlespacing{\subsection}{0pt}{12pt plus 4pt minus 2pt}{6pt plus 2pt minus 2pt}
\titlespacing{\subsubsection}{0pt}{12pt plus 4pt minus 2pt}{4pt plus 2pt minus 2pt}

% \autoref: Subsubsection umbenennen
\addto\extrasngerman{
    \def\subsubsectionautorefname{Unterabschnitt}
}

% Titel
\title{Bachelorarbeit}

% Autor
\author{Andreas Huber}

% Datum
\date{\today}

%
% Start
% des
% Dokuments
%
\begin{document}

% Titelseite
\thispagestyle{empty}

\begin{figure}[t]
 \centering
 \includegraphics[width=0.4\textwidth]{assets/oth/logo}
\end{figure}

\begin{verbatim}
\end{verbatim}

\begin{center}
    \Large{Ostbayerische Technische Hochschule Regensburg} \\
    \Large{Fakultät für Informatik und Mathematik}
\end{center}

\begin{verbatim}
\end{verbatim}

\begin{center}
    \doublespacing
    \textbf{\huge{\titleDocument}}\\

    \onehalfspacing

    % \begin{center}
    %     Zur Erlangung des akademischen Grades \\ Bachelor of Science (B. Sc.)
    % \end{center}

    \begin{verbatim}
    \end{verbatim}

    \begin{doublespace}
        \textbf{\Large{{~\subjectDocument}}}
    \end{doublespace}
\end{center}

\vspace*{\fill}

\begin{flushleft}
    \begin{tabularx}{\linewidth}{@{}>{\bfseries}l@{\hspace{.9em}}X@{}}
        \textbf{Vorgelegt von:} & Andreas Huber <andreas.huber@st.oth-regensburg.de> \\
        \textbf{Matrikelnummer:}& 3370380 \\
        \textbf{Studiengang:}   & Master Informatik (Schwp. Software Engineering) \\
                                & \\
        \textbf{Betreuer:}      & Prof. Dr. techn., Dipl.-Ing. Markus Kucera \\
                                & \\
        \textbf{Abgabedatum:}   & \today \\
    \end{tabularx}
\end{flushleft}
\newpage

% Römische Nummerierung
\pagenumbering{Roman}

% Inhaltsverzeichnis
\tableofcontents

% Abbildungsverzeichnis
\newpage
\phantomsection
\addcontentsline{toc}{section}{Abbildungsverzeichnis}
\renewcommand\refname{Abbildungsverzeichnis}

\listoffigures

% Tabellenverzeichnis
\newpage
\phantomsection
\addcontentsline{toc}{section}{Tabellenverzeichnis}
\renewcommand\refname{Tabellenverzeichnis}

\listoftables

% Abkürzungsverzeichnis
\newpage
\phantomsection
\section*{Abkürzungsverzeichnis}
\addcontentsline{toc}{section}{Abkürzungsverzeichnis}

\begin{acronym}[Bash]
    % \acro{bash}[Bash]{Bourne-again shell}
\end{acronym}

% Arabische Seitennummerierung ab hier
\newpage
\pagenumbering{arabic}

% Einleitung
\begin{abstract}
    Diese Dokumentation enth"alt eine sortierte Liste der wichtigsten
    \LaTeX--Befehle. Die einzelnen Listeneintr"age sind untereinander
    durch viele Querverweise verkettet, die ein Auffinden inhaltlich
    zusammengeh"origer Informationen erheblich erleichtern.
\end{abstract}

\section{Einleitung}\label{einleitung}
Visuelle Ähnlichkeitsbestimmung von Bildern spielt in der heutigen Zeit eine
immer wichtigere Rolle. Anders als bei der Klassifikation von Bildern, geht es
dabei nicht um die korrekte Klassenzuordnung des Bildes, sondern um die konkrete 
visuelle Ähnlichkeit zwischen zweier Bilder \parencite{intro-classification}. 
Inhaltsbasierte Bildvergleichsalgorithmen berücksichtigen Parameter, wie Farbe,
Form, Textur und Struktur. Aus diesen Informationen wird die Überschneidung und
somit die Ähnlichkeit zweier Bilder bestimmt \parencite{intro-cibr}.

Inhaltsbasierte Algorithmen werden neben der umgekehrten Bildersuche
auch beispielsweise für Dublettenerkennung, Anwendung des
Urheberrechts-Diensteanbieter-Gesetzes, Qualitätsprüfung und vieles mehr
verwendet. Da diese Problematik noch immer ein offenes Forschungsthema ist und
es bisher keine allgemeingültige Lösung gibt, beschäftigt sich dieses Paper mit
einer Übersicht aller aktuell gängigen Kategorien an inhaltsbasierten
Bildvergleichsalgorithmen.

\section{Algorithmen zur Ähnlichkeitsbestimmung zweier Bilder}
\subsection{Kategorisch durch Schlüsselwörter}
\subsection{Inhaltsbasierte Algorithmen}
In den folgenden Unterkapiteln stellt Bild A jeweils das Suchbild und Bild B
das Referenzbild dar.
\newpage
\subsubsection{Perceptual Hashing}
Eine der einfachsten, aber auch anfälligsten Methoden zur Bestimmung der
Ähnlichkeit von zwei Bildern ist das Hashing. Dabei gibt es viele verschiedene
Ansätze und Vorgehensmodelle. Die gängigste Methode ist die Anwendung des
pHashes, auch genannt Perceptual Hashing.

Beim Perceptual Hashing werden sowohl für Bild A als auch für Bild B ein Hash
berechnet. Die daraus resultierende Hamming-Distanz zwischen den Hashes ergibt
die Ähnlichkeit der Bilder. Je geringer die Distanz, desto ähnlicher. Das
Verfahren ist nicht normiert und Stand \today{} noch ein offenes
Forschungsthema.

Im Folgenden wird ein beispielhaftes pHash-Verfahren erläutert. Zuerst wird
sowohl das Referenz- als auch das Suchbild in eine Graustufen-Grafik
umgewandelt. Schließlich wird die Grafik auf 32x32 Pixel skaliert. Auf das
entstandene Grauwertbild folgen zwei Diskrete Kosinus Transformationen (1. Pro
Zeile, 2. Pro Spalte). Die hochfrequenten Abschnitte befinden sich nun links
oben in einer 8x8 Matrix. Daraufhin wird der Median-Grauwert der 64 Pixel
berechnet. Jeder Pixel, dessen Grauwert unter dem Durchschnitt liegt wird weiß
eingefärbt, der Rest schwarz. Daraus ergibt sich ein 64-Bit langer Hashwert
(schwarz: 0, weiß: 1). Zuletzt wird durch eine Subtraktion der beiden
generierten Bitfolgen die Hamming-Distanz berechnet. Mittels einem vorher
festgelegten Schwellenwert wird jetzt auf Basis des Hamming-Abstands bestimmt,
ob die Bilder als ähnlich eingestuft werden.

\begin{itemize}
    \item [\textbf{Vorteile}]
    \item Schnelle Performance, wenn die Hashes der Referenzbilder bereits in
    der Datenbank liegen
    \item Einfach zu implementieren
    \item Keine Trainingsdaten nötig
    \item Robust gegen Wasserzeichen, Farbfilter, leichte Helligkeits- und
    Kontraständerungen, Gammakorrekturen, Skalierungen sowie Komprimierungen
\end{itemize}

\begin{itemize}
    \item [\textbf{Nachteile}]
    \item Nicht robust gegen Spiegelungen, Rotierungen und Verzerrungen
    \item Nicht robust gegen Zuschneidungen, neu eingefügten Elementen oder
    Änderungen des Blickwinkels
\end{itemize}

\newpage

\subsubsection{Histogram Intersection}
Ein robusteres, aber auch rechenaufwendigeres Verfahren ist der Histogram
Intersection Algorithmus von Swain und Ballard. Histogramme sind stabil
gegenüber Translation und Rotation um die Betrachtungsachse. Außerdem ändern
sie sich bei Änderungen des Blickwinkels, des Maßstabs und bei Verdeckung von
Elementen nur langsam. Gegen Belichtungsänderungen können stark reduzierte
Farbhistogramme jedoch Probleme bereiten. 
% Quelle: Swain and Ballard P.3

Im Standardfall wird die Histogram Intersection auf Farbhistogramme angewandt. 
Zuerst muss die Menge an zu diskretisierenden Farben im Histogramm festgelegt
werden - zum Beispiel 100. Folglich gibt es 100 mögliche \glqq{}Farbeimer\grqq{}
in die wir jeden Pixel der Bilder einsortieren. Mit den durch das Referenz- und
das Suchbild berechneten Histogrammen, kann schließlich die Überschneidung
festgestellt werden. Je stärker sich die Histogramme überschneiden, desto
ähnlicher sind die Bilder. Zusätzlich wird wie beim Hashing ein gewisser
Schwellenwert vorher definiert. Der Vorgang kann beispielsweise auch auf die
einzelnen Farbkanäle aufgeteilt werden. Bei RGB würde das bedeuten, dass man
jeweils den Rot-, Grün-, und Blau-Kanal einzeln betrachtet. 

Weitere Untersuchungen haben ergeben, dass sowohl die Wahl des Farbraums als
auch die Festlegung des Quantization Levels (Anzahl der Farbeimer) eine große
Rolle bei der Erfolgsschance dieses Vorgehensmodells spielen. Hierbei sorgte
wohl der CIELab-Farbraum im Allgemeinen für die besten Ergebnisse.

Eine zusätzliche Möglichkeit die Genauigkeit des Algorithmus zu verbessern ist
das Hinzufügen eines sogenannten \glqq{}Texture Direction\grqq{}-Histogramms. In
diesem Fall wird durch eine Eckenerkennung die Struktur und Richtungen des
Bildes bestimmt. Diese werden wie bei den Farbhistogrammen auch in einem
Histogramm mit vorher definierten Bins einsortiert. Abschließend kann, wie oben,
beschrieben die Überschneidung und somit die Ähnlichkeit der Struktur zweier
Bilder mithilfe des \glqq{}Histogram Intersection\grqq{}-Algorithmus von Swain
und Ballard berechnet werden. Offensichtlich wird jedoch jeder weitere 
zusätzlicher Berechnungsschritt die allgemeine Performance des Systems
beeinträchtigen.

\newpage

\subsubsection{Scale-Invariant Feature Transform (SIFT)}
David G. Lowe hat mit der Einführung des SIFT-Algorithmus einen für die
Bildverarbeitung wertvollen Beitrag geleistet. Der SIFT-Algoritmus findet lokale
Merkmale in Bildern. Diese Merkmale sind invariant gegenüber Bildskalierung und
-drehung und teilweise invariant gegenüber Änderungen der Beleuchtung sowie des
Betrachtungswinkels. Die Wahrscheinlichkeit einer Störung durch Verdeckung
einzelner Elemente oder auftretenden Bildrauschens wird, durch räumliche
Streuung und einem vielseitigen Frequenzbereich, erreicht. Im Folgenden wird die
grobe Vorgehensweise zur Findung der lokalen Merkmale erläutert.

\begin{enumerate}
    \item \textbf{Scale-space extrema detection}\newline
    Zu Beginn wird aus einem Bild durch repetetives Unschärfen (mittels
    Gaußschen Unschärfefilter) und Halbieren der Größe eine Serie an Bildern
    erzeugt. Auf die Bilder der Bilderserie folgt die Anwendung der Gaußschen
    Differenzfunktion.
    \item \textbf{Keypoint localization}\newline
    Aus den vorher differenzierten Grafiken werden jetzt mit dem
    Marr-Hildreth-Operator interessante Schlüsselpunkte bzw. Merkmale 
    herausgefiltert. Eine dem Harris-Corner Detector ähnliche Prozedur wird
    angewandt, um Punkte auf Linien zwischen Ecken und Punkte mit wenig
    Kontrast (uninteressante Punkte) herauszusieben.
    \item \textbf{Orientation assignment}\newline
    Mithilfe eines Histogramms kann den Schlüsselpunkten nun eine Orientierung
    zugewiesen werden. Dadurch werden die Punkte neben der Robustheit gegenüber
    Skalierung und Translation auch gegen Rotation invariant.
    \item \textbf{Keypoint descriptor}\newline
    Zuletzt wird nach Anwendung einiger mathematischer Verfahren eine Art
    eindeutiger Fingerabdruck des Merkmals berechnet. Schließlich wird dadurch
    die Stabilität gegenüber begrenzter lokaler Formverzerrungen und
    Beleuchtungsänderungen erreicht.
\end{enumerate}

Nach Anwendung dieses Algorithmus muss man bei einem 500x500 Pixel großen Bild
mit in etwa 2000 robusten Merkmalen rechnen. Werden die gefundenen Merkmale in
einer Datenbank gespeichert, können sie schließlich in nahezu Echtzeit auf die
Schlüsselpunkte eines Suchbilds verglichen werden.

\subsubsection{Wavelet Transformation}
\subsubsection{DeepRanking}

% Fazit
\newpage
\section{Fazit und Ausblick}\label{fazit-u-ausblick}

% Römische Nummerierung
\newpage
%\pagenumbering{Roman}
%\setcounter{page}{5}

% Literaturliste soll im Inhaltsverzeichnis auftauchen
\newpage
\phantomsection
\addcontentsline{toc}{section}{Quellenverzeichnis}
% Literaturverzeichnis anzeigen
\renewcommand\refname{Quellenverzeichnis}
\printbibliography
% \bibliography{assets/literatur}

% Anhang
\newpage
\phantomsection
\appendix
\section*{Anhang}
\markboth{Anhang}{}
\addcontentsline{toc}{section}{Anhang}

\end{document}