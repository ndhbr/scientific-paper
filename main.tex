% Festlegung des allgemeinen Dokumentenformats
\documentclass[a4paper,12pt]{article}

% Schrift
\usepackage[T1]{fontenc}
\usepackage{lmodern}
\usepackage[utf8]{inputenc}
\usepackage[ngerman]{babel}

% Bilder
\usepackage{graphicx}
\usepackage{float}
\graphicspath{{./assets/img}}

% Variablen
% Variablen
\newcommand{\titleDocument}{Inhaltsbasierte Bildvergleiche}
\newcommand{\subjectDocument}{Möglichkeiten zur Bestimmung der visuellen
Ähnlichkeit zwischen zwei Bildern}
\newcommand*{\bildquelle}{
  \footnotesize Quelle:
}
\newcommand{\code}[1]{\noindent\ignorespaces\texttt{#1}}

% Code
\usepackage{listings}
\usepackage{color}

\definecolor{black}{rgb}{0,0,0}
\definecolor{dkgreen}{rgb}{0,0.6,0}
\definecolor{gray}{rgb}{0.5,0.5,0.5}
\definecolor{mauve}{rgb}{0.58,0,0.82}

\lstset{literate=%
    {Ö}{{\"O}}1
    {Ä}{{\"A}}1
    {Ü}{{\"U}}1
    {ß}{{\ss}}1
    {ü}{{\"u}}1
    {ä}{{\"a}}1
    {ö}{{\"o}}1
    {~}{{\textasciitilde}}1
}

\lstdefinestyle{Bash} {
  frame=single,
  language=Bash,
  aboveskip=3mm,
  belowskip=3mm,
  showstringspaces=false,
  columns=flexible,
  basicstyle={\small\ttfamily},
  numbers=none,
  numberstyle=\tiny\color{black},
  keywordstyle=\color{black},
  commentstyle=\color{black},
  stringstyle=\color{black},
  breaklines=true,
  breakatwhitespace=true,
  tabsize=3
}

\lstset{emph={\$},emphstyle=\textbf}

\lstdefinestyle{Python} {
  frame=single,
  language=Bash,
  aboveskip=3mm,
  belowskip=3mm,
  showstringspaces=false,
  columns=flexible,
  basicstyle={\small\ttfamily},
  numbers=left,
  numberstyle=\tiny\color{mauve},
  keywordstyle=\color{mauve},
  commentstyle=\color{dkgreen},
  stringstyle=\color{dkgreen},
  breaklines=true,
  breakatwhitespace=true,
  tabsize=2
}

% mehrseitige Tabellen ermöglichen
\usepackage{longtable}
\usepackage{diagbox}

% Packet für Seitenrandabstände und Einstellung für Seitenränder
\usepackage{geometry}
% Internet
%\geometry{left=3.5cm, right=2.5cm, top=2.5cm, bottom=2cm}
% Kern
\geometry{a4paper, top=27mm, left=20mm, right=20mm, bottom=35mm, headsep=10mm,
footskip=12mm}

% bricht lange URLs "schön" um
\usepackage[hyphens,obeyspaces,spaces]{url}

% Festlegung Art der Zitierung
\usepackage{csquotes}
\usepackage[style=apa, backend=biber]{biblatex}
\addbibresource{./assets/literatur.bib}

% Abstand zwischen Absätze
\setlength{\parindent}{2em}
\setlength{\parskip}{1em}

% Paket für Zeilenabstand
\usepackage{setspace}
\onehalfspacing
% Kern:
% \setstretch{1.15}

% Um Itemize zu konfigurieren (kein Abstand zur Zeile davor)
\usepackage{enumitem}

% für Bildbezeichner
\usepackage{capt-of}

% für Stichwortverzeichnis
\usepackage{makeidx}

% Für Phantomsection
\usepackage{hyperref}

% Für Tabellen
\usepackage{tabularx}

% Konfiguriere das Inhaltsverzeichnis
\usepackage{tocbasic}

% Anhang: PDF einfügen
\usepackage{pdfpages}

% subsubsubsection durch paragraph
\usepackage{titlesec}
\setcounter{secnumdepth}{4}
\titleformat{\paragraph}
{\normalfont\normalsize\bfseries}{\theparagraph}{1em}{}
\titlespacing*{\paragraph}
{0pt}{3.25ex plus 1ex minus .2ex}{1.5ex plus .2ex}
% Kern:
\titlespacing{\section}{0pt}{12pt plus 4pt minus 2pt}{8pt plus 2pt minus 2pt}
\titlespacing{\subsection}{0pt}{12pt plus 4pt minus 2pt}{6pt plus 2pt minus 2pt}
\titlespacing{\subsubsection}{0pt}{12pt plus 4pt minus 2pt}{4pt plus 2pt minus 2pt}

% \autoref: Subsubsection umbenennen
\addto\extrasngerman{
    \def\subsubsectionautorefname{Unterabschnitt}
}

% Titel
\title{Wissenschaftliche Abhandlung}

% Autor
\author{Andreas Huber}

% Datum
\date{\today}

%
% Start
% des
% Dokuments
%
\begin{document}

% Titelseite
\thispagestyle{empty}

\begin{figure}[t]
 \centering
 \includegraphics[width=0.4\textwidth]{assets/oth/logo}
\end{figure}

\begin{verbatim}
\end{verbatim}

\begin{center}
    \Large{Ostbayerische Technische Hochschule Regensburg} \\
    \Large{Fakultät für Informatik und Mathematik}
\end{center}

\begin{verbatim}
\end{verbatim}

\begin{center}
    \doublespacing
    \textbf{\huge{\titleDocument}}\\

    \onehalfspacing

    % \begin{center}
    %     Zur Erlangung des akademischen Grades \\ Bachelor of Science (B. Sc.)
    % \end{center}

    \begin{verbatim}
    \end{verbatim}

    \begin{doublespace}
        \textbf{\Large{{~\subjectDocument}}}
    \end{doublespace}
\end{center}

\vspace*{\fill}

\begin{flushleft}
    \begin{tabularx}{\linewidth}{@{}>{\bfseries}l@{\hspace{.9em}}X@{}}
        \textbf{Vorgelegt von:} & Andreas Huber <andreas.huber@st.oth-regensburg.de> \\
        \textbf{Matrikelnummer:}& 3370380 \\
        \textbf{Studiengang:}   & Master Informatik (Schwp. Software Engineering) \\
                                & \\
        \textbf{Betreuer:}      & Prof. Dr. techn., Dipl.-Ing. Markus Kucera \\
                                & \\
        \textbf{Abgabedatum:}   & \today \\
    \end{tabularx}
\end{flushleft}
\newpage

% Römische Nummerierung
\pagenumbering{Roman}

% Inhaltsverzeichnis
\tableofcontents

% Abbildungsverzeichnis
\newpage
\phantomsection
\addcontentsline{toc}{section}{Abbildungsverzeichnis}
\renewcommand\refname{Abbildungsverzeichnis}

\listoffigures

% Arabische Seitennummerierung ab hier
\newpage
\pagenumbering{arabic}

% Einleitung
\begin{abstract}
    Diese Dokumentation enth"alt eine sortierte Liste der wichtigsten
    \LaTeX--Befehle. Die einzelnen Listeneintr"age sind untereinander
    durch viele Querverweise verkettet, die ein Auffinden inhaltlich
    zusammengeh"origer Informationen erheblich erleichtern.
\end{abstract}

\section{Einleitung}\label{einleitung}
Visuelle Ähnlichkeitsbestimmung von Bildern spielt in der heutigen Zeit eine
immer wichtigere Rolle. Anders als bei der Klassifikation von Bildern, geht es
dabei nicht um die korrekte Klassenzuordnung des Bildes, sondern um die konkrete 
visuelle Ähnlichkeit zwischen zweier Bilder \parencite{intro-classification}. 
Inhaltsbasierte Bildvergleichsalgorithmen berücksichtigen Parameter, wie Farbe,
Form, Textur und Struktur. Aus diesen Informationen wird die Überschneidung und
somit die Ähnlichkeit zweier Bilder bestimmt \parencite{intro-cibr}.

Inhaltsbasierte Algorithmen werden neben der umgekehrten Bildersuche
auch beispielsweise für Dublettenerkennung, Anwendung des
Urheberrechts-Diensteanbieter-Gesetzes, Qualitätsprüfung und vieles mehr
verwendet. Da diese Problematik noch immer ein offenes Forschungsthema ist und
es bisher keine allgemeingültige Lösung gibt, beschäftigt sich dieses Paper mit
einer Übersicht aller aktuell gängigen Kategorien an inhaltsbasierten
Bildvergleichsalgorithmen.

\section{Inhaltsbasierte Algorithmen zur Ähnlichkeitsbestimmung zweier Bilder}

% Mittlere quadratische Abweichung
\newpage
\subsection{Mittlere quadratische Abweichung (Mean squared error)}


% Perceptual Hashing
\newpage
\subsection{Perceptual Hashing}
Eine der einfachsten, aber auch anfälligsten Methoden zur Bestimmung der
Ähnlichkeit von zwei Bildern ist das Hashing. Dabei gibt es viele verschiedene
Ansätze und Vorgehensmodelle. Die gängigste Methode ist die Anwendung des
pHashes, auch genannt Perceptual Hashing. \parencite{hashing-apiumhub}

Beim Perceptual Hashing werden sowohl für Bild A als auch für Bild B ein Hash
berechnet. Die daraus resultierende Hamming-Distanz zwischen den Hashes ergibt
die Ähnlichkeit der Bilder. Je geringer die Distanz, desto ähnlicher. Das
Verfahren ist nicht normiert und noch ein offenes
Forschungsthema. \parencite{hashing-phash}

Im Folgenden wird ein beispielhaftes pHash-Verfahren erläutert. Zuerst wird
sowohl das Referenz- als auch das Suchbild in eine Graustufen-Grafik
umgewandelt. Schließlich wird die Grafik auf 32x32 Pixel skaliert. Auf das
entstandene Grauwertbild folgen zwei Diskrete Kosinus Transformationen (1. Pro
Zeile, 2. Pro Spalte). Die hochfrequenten Abschnitte befinden sich nun links
oben in einer 8x8 Matrix. Daraufhin wird der Median-Grauwert der 64 Pixel
berechnet. Jeder Pixel, dessen Grauwert unter dem Durchschnitt liegt wird weiß
eingefärbt, der Rest schwarz. Daraus ergibt sich ein 64-Bit langer Hashwert
(schwarz: 0, weiß: 1). Zuletzt wird durch eine Subtraktion der beiden
generierten Bitfolgen die Hamming-Distanz berechnet. Mittels einem vorher
festgelegten Schwellenwert wird jetzt auf Basis des Hamming-Abstands bestimmt,
ob die Bilder als ähnlich eingestuft werden. \parencite{hashing-apiumhub}

\noindent{\textbf{Vorteile}}
\begin{itemize}[topsep=0pt]
    \item Schnelle Performance, wenn die Hashes der Referenzbilder bereits in
    einer dafür geeigneten Datenstruktur (z.B. k-d-Bäume, VP Bäume oder
    Kugelbäume) vorliegen \parencite{hashing-lvngd}
    \item Einfach zu implementieren
    \item Keine Trainingsdaten nötig
    \item Robust gegen Wasserzeichen, Farbfilter, leichte Helligkeits- und
    Kontraständerungen, Gammakorrekturen, Skalierungen sowie Komprimierungen
    \parencite{hashing-phash}
\end{itemize}

\noindent{\textbf{Nachteile}}
\begin{itemize}[topsep=0pt]
    \item Nicht robust gegen Spiegelungen, Rotierungen und Verzerrungen
    \item Nicht robust gegen Zuschneidungen, neu eingefügten Elementen oder
    Änderungen des Blickwinkels
\end{itemize}

% Histogram Intersection
\newpage
\subsection{Histogram Intersection}
Ein robusteres, aber auch rechenaufwendigeres Verfahren ist der Histogram
Intersection Algorithmus von Swain und Ballard. Farbhistogramme sind stabil
gegenüber Spiegelungen, Verschiebungen und leichten Rotationen um die
Betrachtungsachse (siehe Abbildung \ref{fig:hintersection}). Außerdem ändern sie
sich bei Änderungen des Blickwinkels, des Maßstabs und bei Verdeckung von
Elementen nur langsam. Gegen Belichtungsänderungen können stark reduzierte
Farbhistogramme jedoch Probleme bereiten. \parencite{histogram-swain-ballard}

Im Standardfall wird die Histogram Intersection auf Farbhistogramme angewandt. 
Zuerst muss die Menge an zu diskretisierenden Farben im Histogramm festgelegt
werden - zum Beispiel 200. Folglich gibt es 200 mögliche \glqq{}Farbeimer\grqq{}
(Bins) in die jeder Pixel der Bilder einsortiert wird. Nach dem
Übereinanderlegen der durch das Referenz- und das Suchbild berechneten
Histogrammen, wird schließlich deren Überschneidung berechnet. Je stärker sich
die Histogramme überschneiden, desto ähnlicher sind die Bilder
\parencite{histogram-image-similarity}. Zusätzlich wird wie beim Hashing ein
gewisser Schwellenwert vorher definiert. Der Vorgang kann beispielsweise auch
auf die einzelnen Farbkanäle aufgeteilt werden. Bei RGB würde das bedeuten, dass
man jeweils den Rot-, Grün-, und Blau-Kanal einzeln betrachtet.
\parencite{histogram-swain-ballard}

Eine weitere Untersuchung hat ergeben, dass sowohl die Wahl des Farbraums als
auch die Festlegung des Quantization Levels (Anzahl der Bins) eine große Rolle
bei der Erfolgsschance dieses Vorgehensmodells spielen. In dieser Studie
lieferte der CIELab-Farbraum im Allgemeinen die besten Ergebnisse.
\parencite{histogram-image-similarity}

Um die Genauigkeit des Algorithmus zusätzlich zu verbessern, können neben
Farbhistogrammen weitere Histogramme erstellt und verglichen werden. Ein
mögliches Beispiel wäre ein \glqq{}Texture Direction\grqq{} oder
\glqq{}Texture Scale\grqq{} Histogramm. In diesen Fällen wird nicht versucht
anhand der Farben eine Ähnlichkeit zu erkennen, sondern durch die Struktur des
Bildes. Hier werden beispielsweise aus dem Bild extrahierte Ecken oder Kanten in
ein Histogramm sortiert und schließlich auf Überschneidungen analysiert
\parencite{histogram-stackoverflow}. Natürlich wird jedoch jeder weitere
zusätzlicher Berechnungsschritt die allgemeine Performance des Systems
beeinträchtigen.

\begin{figure}[H]
    \centering
    \includegraphics[width=\textwidth]{histogram-intersection}
    \caption{Histogram Intersection: Anwendung an verschiedenen Testbildern}
    \label{fig:hintersection}
    \bildquelle{Eigene Darstellung}
\end{figure}

% Scale-Invariant Feature Transform (SIFT)
\newpage
\subsection{Scale-Invariant Feature Transform (SIFT)}
David G. Lowe hat mit der Einführung des SIFT-Algorithmus einen für die
Bildverarbeitung wertvollen Beitrag geleistet. Der SIFT-Algoritmus findet lokale
Merkmale in Bildern. Diese Merkmale sind invariant gegenüber Bildskalierung und
-drehung und teilweise invariant gegenüber Änderungen der Beleuchtung sowie des
Betrachtungswinkels. Die Wahrscheinlichkeit einer Störung durch Verdeckung
einzelner Elemente oder auftretenden Bildrauschens wird, durch räumliche
Streuung und einem vielseitigen Frequenzbereich, erreicht. Im Folgenden wird die
grobe Vorgehensweise zur Findung der lokalen Merkmale erläutert.
\parencite{sift-distinctive-features}

\begin{enumerate}
    \item \textbf{Scale-space extrema detection}\newline
    Zu Beginn wird aus einem Bild durch repetetives Unschärfen (mittels
    \textit{Gaußschen Weichzeichner}\footnote{https://datacarpentry.org/image-processing/06-blurring/ [09.12.2022]})
    und Halbieren der Größe eine Serie an Bildern erzeugt (siehe Abbildung
    \ref{fig:sift1}). Auf die Bilder der Bilderserie folgt die Anwendung der
    \textit{Gaußschen Differenzfunktion}\footnote{https://www.sciencedirect.com/science/article/pii/B9780123694072500097 [09.12.2022]}
    (siehe Abbildung \ref{fig:sift2}). \parencite{sift-web-scale-space}

    \begin{figure}[H]
        \centering
        \includegraphics[width=13.3cm]{sift-1}
        \caption{SIFT: Erzeugung der Bilderserie}
        \label{fig:sift1}
        \bildquelle{Eigene Darstellung; http://weitz.de/sift/}
    \end{figure}

    \begin{figure}[H]
        \centering
        \includegraphics[width=13.3cm]{sift-2}
        \caption{SIFT: Anwendung Gaußscher Differenzfunktion}
        \label{fig:sift2}
        \bildquelle{Eigene Darstellung; http://weitz.de/sift/}
    \end{figure}

    \item \textbf{Keypoint localization}\newline
    Aus den vorher differenzierten Grafiken werden mit dem
    \textit{Marr-Hildreth-Operator}\footnote{https://theailearner.com/2019/05/25/laplacian-of-gaussian-log [09.12.2022]}
    interessante Schlüsselpunkte bzw. Merkmale herausgefiltert (siehe Abbildung
    \ref{fig:sift3}) \parencite{sift-web-keypoint}. Eine dem \textit{Harris-Corner-Detector}\footnote{https://docs.opencv.org/3.4/dc/d0d/tutorial\_py\_features\_harris.html [09.12.2022]}
    ähnliche Prozedur wird angewandt, um unwichtige Punkte herauszusieben
    (siehe Abbildung \ref{fig:sift5}) \parencite{sift-web-low-contrast}.

    \begin{figure}[H]
        \centering
        \begin{subfigure}{.5\textwidth}
          \centering
          \includegraphics[width=.75\linewidth]{sift-3}
          \caption{Extrahierte Schlüsselpunkte}
          \label{fig:sift3}
        \end{subfigure}%
        \begin{subfigure}{.5\textwidth}
          \centering
          \includegraphics[width=.75\linewidth]{sift-5}
          \caption{Filterung unwichtiger Schlüsselpunkte}
          \label{fig:sift5}
        \end{subfigure}
        \caption{SIFT: Schlüsselpunkt-Lokalisierung}
        \label{fig:sift-keypoint}
        \bildquelle{Eigene Darstellung; http://weitz.de/sift/}
    \end{figure}

    \item \textbf{Orientation assignment}\newline
    Mithilfe eines Histogramms kann den Schlüsselpunkten nun eine Orientierung
    zugewiesen werden (siehe Abbildung \ref{fig:sift-orientation}). Dadurch
    werden die Punkte neben der Robustheit gegenüber Skalierung und Verschiebung
    auch gegen Rotation invariant (siehe Abbildung \ref{fig:sift}).
    \parencite{sift-web-orientation}

    \begin{figure}[H]
        \centering
        \begin{subfigure}{.5\textwidth}
          \centering
          \includegraphics[width=.5\linewidth]{sift-6.png}
          \caption{Orientierung des Merkmals}
          \label{fig:sift6}
        \end{subfigure}%
        \begin{subfigure}{.5\textwidth}
          \centering
          \includegraphics[width=.5\linewidth]{sift-6b.png}
          \caption{Zugehöriges Farbhistogramm (normalisiert)}
          \label{fig:sift6b}
        \end{subfigure}
        \caption{SIFT: Zuordnung der Orientierung}
        \label{fig:sift-orientation}
        \bildquelle{Eigene Darstellung; http://weitz.de/sift/}
    \end{figure}

    \item \textbf{Keypoint descriptor}\newline
    Zuletzt wird nach Anwendung weiterer mathematischer Verfahren eine Art
    Fingerabdruck des Merkmals berechnet (siehe Abbildung \ref{fig:sift7}).
    Dadurch wird die Stabilität gegenüber begrenzter lokaler Formverzerrungen
    und Beleuchtungsänderungen erreicht (siehe Abbildung \ref{fig:sift}).
    \parencite{sift-web-descriptor}

    \begin{figure}[H]
        \centering
        \includegraphics[width=5cm]{sift-7}
        \caption{SIFT: Pro Schlüsselpunkt erzeugte Deskriptoren (Fingerabdruck)}
        \label{fig:sift7}
        \bildquelle{Eigene Darstellung; http://weitz.de/sift/}
    \end{figure}
\end{enumerate}

Nach Anwendung dieses Algorithmus sollte man bei einem 500x500 Pixel großen Bild
mit rund 2000 robusten Merkmalen rechnen. Werden die gefundenen Merkmale in
einer Datenbank gespeichert, können sie in nahezu Echtzeit auf die
Schlüsselpunkte eines Testbilds verglichen werden.
\parencite{sift-distinctive-features}

Ein ähnlicher, jedoch bis 2033 patentierter\footnote{https://patents.google.com/patent/CN103640018A/en [09.12.2022]}, Algorithmus zur Erkennung von Bildmerkmalen ist der
\glqq{}Speeded up robust features (SURF)\grqq{} \parencite{sift-surf}. Ferner
gibt es viele ähnliche frei verfügbare Algorithmen der Gruppe
\glqq{}Bildmerkmalsfindung\grqq{}.

\begin{figure}[H]
    \centering
    \includegraphics[width=\textwidth]{sift}
    \caption{SIFT: Anwendung an verschiedenen Testbildern}
    \label{fig:sift}
    \bildquelle{Eigene Darstellung}
\end{figure}


% Structural Similarity Index (SSIM)
\newpage
\subsection{Structural Similarity Index (SSIM)}
Anders als die meisten anderen Vorgehensmodelle versucht die menschliche
Wahrnehmung nicht Fehler zu Quantifizieren, sondern strukturelle Informationen
aus einer Szene wiederzuerkennen. Um dieses Verhalten nachzuahmen, haben
Forschende Mitglieder der IEEE den Structual Similarity Index entworfen.

\noindent{Der Index extrahiert, vergleicht und kombiniert folgende drei
Parameter:}
\begin{itemize}[topsep=0pt]
    \item Leuchtdichte
    \item Kontrast
    \item Struktur
\end{itemize}

Im ersten Schritt wird die Leuchtdichte durch Mittelwertbildung über alle
Pixelwerte gemessen. Anschließend wird der Kontrast durch Berechnung der
Standardabweichung aller Werte ermittelt. Vereinfacht dargestellt wird die
Struktur durch die Teilung des Eingangssignals durch seine Standardabweichung
geteilt, sodass eine genormte Standardabweichung entsteht.

Sobald die drei Werte berechnet wurden, werden diese anhand, im Original-Paper,
veröffentlichten mathematischen Funktionen verglichen und schließlich
kombiniert.

Obendrein sei es nützlich den Index nicht global auf das Bild anzuwenden. Das
Paper gibt an den Algorithmus vorzugsweise auf verschiedene lokale Stellen
einzusetzen. Erstens seien für die Analyse interessante Objekte oft instationär
und zweitens können Bildverzerrungen auch räumlich variabel sein. Um die
Referenz zur Nachahmung der menschlichen Wahrnehmung zu ziehen, ist ebenfalls
festzustellen, dass Menschen ebenfalls nur lokale Bereiche einer Grafik in hoher
Auflösung wahrnehmen können.

Auch wenn der SSIM durch seine Einfachheit, gegenüber anderen Verfahren, eine
überragende Performance aufweist, ist der Algorithmus sehr anfällig für
Skalierungen, Translationen und Rotierungen.

Aufbauend auf den SSIM-Algorithmus gibt es weitere optimierende Forschungen, wie
zum Beispiel der Complex Wavelet Structural Similarity Index. Der CW-SSIM
adressiert die vorher genannten Probleme und erzeugt durch Verwendung von
Wavelet-Transformationen eine Robustheit gegenüber kleinen Translationen und
Rotierungen. \parencite{ssim-complex-wavelet}

% Fazit
\newpage
\section{Fazit und Ausblick}\label{fazit-u-ausblick}

% Römische Nummerierung
\newpage
%\pagenumbering{Roman}
%\setcounter{page}{5}

% Literaturliste soll im Inhaltsverzeichnis auftauchen
\newpage
\phantomsection
\addcontentsline{toc}{section}{Quellenverzeichnis}
% Literaturverzeichnis anzeigen
\renewcommand\refname{Quellenverzeichnis}
\printbibliography
% \bibliography{assets/literatur}

% Anhang
\newpage
\phantomsection
\appendix
\section*{Anhang}
\markboth{Anhang}{}
\addcontentsline{toc}{section}{Anhang}

\end{document}